\chapter{緒論}
\section{研究動機}
{
瀏覽網頁對於大部分的人已經成為日常生活中不可缺少的行為,而瀏覽網頁所留下的歷史紀錄能夠描述出每一位使用者的瀏覽偏好。起初,我以最直觀的角度對網頁瀏覽歷史紀錄進行各方面的分析後成功得到使用者的興趣喜好、休閒娛樂、工作領域相關資訊。在分析的過程中,我對使用者之網頁瀏覽紀錄資料進行多次的資料前處理(Data preprocessing),讓我更加了解使用者之網頁瀏覽行為。由於所處研究單位擁有較為完整的 $600$ 位使用者之網頁瀏覽紀錄以及其個人資訊,其中個人資訊包括:性別、年齡、感情狀態與大六性格特質分數~\cite{greaves2015regional},也激發了我對網頁瀏覽歷史紀錄有更多不同的研究想法。\par
基於此資料集擁有詳細的使用者個人資訊,我認為可以藉由此資料集對網頁瀏覽歷史紀錄做更深度的分析,如果能夠藉由使用者之網頁瀏覽歷史紀錄來預測使用者之個人資訊甚至預測使用者之個性,將得以解決使用者之詳細個人資訊難以取得的問題。
}

\section{研究目標}
{
本篇論文之主要目標為透過使用者之網頁瀏覽紀錄預測使用者之詳細個人資訊以及大六性格特質測驗分數,由於取得使用者之詳細個人資訊必須花費使用者三至五分鐘填寫個人資訊表單,而更深入的使用者大六性格特質測驗分數則必須花費三十分鐘以上。最困難的部分在於大部分使用者是不願意花費如此大量的時間來填寫問卷。藉此我認為藉由使用者的網頁瀏覽歷史紀錄來預測使用者之個人資訊以及大六性格特質的測驗分數將能夠大幅降低測驗成本以及使用者填寫個人詳細資訊、測驗時間。基於上述原因,本篇論文之研究目標為以下四點:
\begin{enumerate}
\item 對使用者之網頁瀏覽歷史紀錄進行資料前處理,並尋找主要代表使用者個性之特徵。
\item 利用使用者網頁瀏覽歷史紀錄進行使用者之詳細個人資訊預測。
\item 利用使用者網頁瀏覽歷史紀錄進行使用者之大六性格特質測驗分數預測。
\item 透過分群(Clustering)結合監督式學習(Supervise learning)提升預測使用者詳細個人資訊與大六性格特質測驗分數之預測效果。
\end{enumerate}
}

\section{研究貢獻}
{
由於部分單位之伺服器端雖然擁有每一位使用者之網頁瀏覽紀錄,但是並不包含使用者之個人資訊,例如:網頁管理單位等。透過本篇論文之研究成果,藉由使用者之網頁瀏覽歷史紀錄來預測使用者詳細個人資訊,將得以改善上述問題。\par

將本篇論文之研究成果應用在其他領域也能夠得到相當高的利益價值。其中最具有代表性的成果為電子商務網站之應用,若電子商務網站在招攬會員時,能夠取得會員之網頁瀏覽歷史紀錄以及相關個人資訊(年齡、性別),或者透過會員在其網頁平台上不同商品的瀏覽紀錄,並進行使用者之性格特質預測,將能夠根據使用者之族群、個性給予不同的銷售手段,得到較高的廣告效果~\cite{luo2016online}。\par

由於每一位使用者之間的相似性與其生活環境、網頁瀏覽習慣具有關聯性,對於相同類型之使用者給予相似的推薦結果可能較為符合推薦對象之喜好,因此將使用者之相關資訊運用於推薦系統上是能夠增加特徵並幫助推薦引擎之效能。
}
\clearpage
\section{論文架構}
{
本篇論文共分為六個章節,其架構如下:\\
\\
第一章、說明本篇論文之研究動機、研究目標、研究貢獻、論文架構。\\
第二章、介紹與本篇論文之類似主題研究與其研究成果。\\
第三章、說明本篇論文所使用的資料集與資料前處理之過程。\\
第四章、說明實驗之設計與步驟,包括特徵選擇與分類器選擇。\\
第五章、展示實驗之多種預測結果,並且進行比較與效能評估。\\
第六章、本篇論文之結論與未來展望。

}